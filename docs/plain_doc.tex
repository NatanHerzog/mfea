\documentclass[a4paper]{article}

\usepackage{amsmath}
\usepackage{breqn}
\usepackage{geometry}
\geometry{
    lmargin = 0.5in,
    rmargin = 0.5in,
    tmargin = 0.5in,
    bmargin = 0.5in
}
\usepackage{parskip}
\setlength{\parindent}{0pt}
\setlength{\parskip}{\baselineskip}
\usepackage{xcolor}
\usepackage{matlab-prettifier}
\AtBeginEnvironment{lstlisting}{\vspace{\baselineskip}}
\AtBeginEnvironment{equation*}{\vspace{-0.5\baselineskip}}
\AtBeginEnvironment{align*}{\vspace{-\baselineskip}}
\AfterEndEnvironment{lstlisting}{\vspace{-\baselineskip}}

\begin{document}

\fontsize{12pt}{13pt}\selectfont

\textit{\textbf{Disclaimer:} There is some underlying derivation that is left out of this document at the moment for the sake of ``brevity.''}

\section{1D Beam Element}

\begin{equation*}
    \Pi = U + \Omega
\end{equation*}

The following expresses the strain energy (energy stored as a result of displacement\textemdash e.g. spring elasticity) for a 1D beam element as a function of the generalized nodal DOF displacements, $\{d\}$, the strain interpolation field, $\left[B(x)\right]$, and the inertial tensor, $\left[D\right]$.

\begin{equation*}
    U = \int_{0}^{L} \dfrac{m\kappa}{2} dx =
    \int_{0}^{L} \dfrac{\left\{\sigma\right\}^{T}\left\{\epsilon\right\}}{2} dx =
    \left\{d\right\}^{T} \left[ \int_{0}^{L} \left[B\right]^{T} \left[D\right] \left[B\right] dx \right] \left\{d\right\}
\end{equation*}

The following expresses the potential energy induced by the presence of a gravitational field and external forces for a 1D beam element. This is a function of the applied distributed loads (e.g. gravity), $b$, the generalized nodal DOF displacements, $\{d\}$, the shape functions, $\left[N(x)\right]$, and the external loads and subsequent displacements, $\left\{f^{C}\right\}$

\begin{equation*}
    \Omega = - \int_{0}^{L} bv dx - \left\{f^{C}\right\}\left\{d\right\} =
    -b \int_{0}^{L} \left[N\right]\left\{d\right\} dx - \left\{f^{C}\right\}\left\{d\right\}
\end{equation*}

The shape functions, $\left[N(x)\right]$, are derived from a general displacement field, $v(x)$, which must be of order $3$ because the 1D beam element has $4$ degrees of freedom ($v_{1}$, $\phi_{1}$, $v_{2}$, $\phi_{2}$). This derivation is as follows:

\begin{equation*}
v(x) = a_{1}x^{3} + a_{2}x^{2} + a_{3}x^{1} + a_{4}x^{0}\\
\end{equation*}

The slope at any location, $\phi(x)$, is defined as follows:

\begin{equation*}
\phi(x) = \dfrac{d}{dx}\left[v(x)\right] = 3a_{1}x^{2} + 2a_{2}x^{1} + a_{3}x^{0} + 0
\end{equation*}

From here, expressions for the values at the two nodes are found:

\begin{align*}
v(x=0) = v_{1} & = a_{1}0 + a_{2}0 + a_{3}0 + a_{4}\\
\phi(x=0) = \phi_{1} & = a_{1}0 + a_{2}0 + a_{3} + a_{4}0\\
v(x=L) = v_{2} & = a_{1}L^{3} + a_{2}L^{2} + a_{3}L + a_{4}\\
\phi(x=L) = \phi_{2} & = a_{1}3L^{2} + a_{2}2L + a_{3} + a_{4}0
\end{align*}

\begin{equation*}
\begin{Bmatrix}
    v_{1} \\ \phi_{1} \\ v_{2} \\ \phi_{2}
\end{Bmatrix} =
\begin{bmatrix}
    0 & 0 & 0 & 1\\
    0 & 0 & 1 & 0\\
    L^{3} & L^{2} & L & 1\\
    3L^{2} & 2L & 1 & 0
\end{bmatrix}
\begin{Bmatrix}
    a_{1} \\ a_{2} \\ a_{3} \\ a_{4}
\end{Bmatrix}
\end{equation*}

Then, returning to the formulation for the generalized displacement field $v(x)$, it can be rewritten in vector multiplication form as follows:

\begin{equation*}
v(x) =
\begin{bmatrix}
    x^{3} & x^{2} & x & 1
\end{bmatrix}
\begin{Bmatrix}
    a_{1} \\
    a_{2} \\
    a_{3} \\
    a_{4}
\end{Bmatrix}
\end{equation*}

The coefficient vector, $\{a\}$, can be substituted for the matrix-vector multiplication expression above (after inverting the matrix):

\begin{equation*}
v(x) =
\underbrace{
    \begin{bmatrix}
        x^{3} & x^{2} & x & 1
    \end{bmatrix}
    \begin{bmatrix}
        0 & 0 & 0 & 1\\
        0 & 0 & 1 & 0\\
        L^{3} & L^{2} & L & 1\\
        3L^{2} & 2L & 1 & 0
    \end{bmatrix}^{-1}
}_{\left[N({x}\right]}
\underbrace{
    \begin{Bmatrix}
        v_{1} \\ \phi_{1} \\ v_{2} \\ \phi_{2}
    \end{Bmatrix}
}_{\{d\}}
\end{equation*}

For simplicity, this is rewritten as follows:

\begin{equation*}
    v(x) = \left[N(x)\right]\{d\}
\end{equation*}

where $\left[N(x)\right]$ is referred to by the name ``Shape Functions'' and the vector $\{d\}$ is the nodal DOF vector.

The Shape Functions provide an expression for evaluating the displacement (both $v$ and $\phi$) throughout the length of the 1D element expressed as a linear combination of the influence from the the two degrees of freedom at each node. Essentially, the Shape Functions translate from the fundamentally discrete nodal solutions produced by the finite element method into a continuous displacement field for evaluating results.

This work assumes linear Shape Functions for simplicity, but the beam element can also be formulated with higher order Shape Functions, which can sometimes be beneficial.

Anyway, using MATLAB to perform the (relatively simple) matrix multiplication, the Shape Functions are found as follows:

\begin{lstlisting}[style=Matlab-Editor, mlunquotedstringdelim={`}{'}, frame=single]
syms `x' `L'                                            % Symbolic x and L
xvec = [x^3, x^2, x, 1];                            % Used in vector notation
M = [0,     0,      0, 1;...
     0,     0,      1, 0;...
     L^3,   L^2,    L, 1;...
     3*L^2, 2*L,    1, 0];                          % Coefficient matrix
N = xvec * inv(M);                                  % Shape functions
\end{lstlisting}

\begin{dmath*}
    \left[N(x)\right] =
    \begin{bmatrix}
        \left(
            \dfrac{2}{L^{3}}x^{3} - \dfrac{2}{L^{2}}x^{2} + 1
        \right) &
        \left(
            \dfrac{1}{L^{2}}x^{3} - \dfrac{2}{L}x^{2} + x
        \right) &
        \left(
            -\dfrac{2}{L^{3}}x^{3} + \dfrac{3}{L^{2}}x^{2}
        \right) &
        \left(
            \dfrac{1}{L^{2}}x^{3} - \dfrac{1}{L}x^{2}
        \right)
    \end{bmatrix}
\end{dmath*}





\begin{lstlisting}[style=Matlab-Editor, mlunquotedstringdelim={`}{'}, frame=single]
B = diff(N, x, 2);                                  % Strain field?
syms `EI'                                             % Symbolic EI
D = EI;                                             % Scalar tensor
K = transpose(int(transpose(B) * D * B, x, 0, L))   % Stiffness matrix
\end{lstlisting}

Angular Displacement Field

I need to do more research into the physical interpretation of this vector.

\begin{dmath*}
    \left[B(x)\right] = \dfrac{\partial^{2} N(x)}{\partial x^{2}} = 
    \begin{bmatrix}
        \left(
            \dfrac{12}{L^{3}}x - \dfrac{4}{L^{2}}
        \right) &
        \left(
            \dfrac{6}{L^{2}}x - \dfrac{4}{L}
        \right) &
        \left(
            -\dfrac{12}{L^{3}}x + \dfrac{6}{L^{2}}
        \right) &
        \left(
            \dfrac{6}{L^{2}}x - \dfrac{2}{L}
        \right)
    \end{bmatrix}
\end{dmath*}

\begin{dmath*}
    \left[K\right] =
    \begin{bmatrix}
        \dfrac{12EI}{L^{3}} & \dfrac{6EI}{L^{2}} & -\dfrac{12EI}{L^{3}} & \dfrac{6EI}{L^{2}} \\[1em]
        \dfrac{6EI}{L^{2}} & \dfrac{4EI}{L} & -\dfrac{6EI}{L^{2}} & \dfrac{2EI}{L} \\[1em]
        -\dfrac{12EI}{L^{3}} & -\dfrac{6EI}{L^{2}} & \dfrac{12EI}{L^{3}} & -\dfrac{6EI}{L^{2}} \\[1em]
        \dfrac{6EI}{L^{2}} & \dfrac{2EI}{L} & -\dfrac{6EI}{L^{2}} & \dfrac{4EI}{L}
    \end{bmatrix}
\end{dmath*}

Combined with the linear truss element for axial stiffness where

\begin{dmath*}
    \left[K\right] =
    \begin{bmatrix}
        \dfrac{EA}{L} & -\dfrac{EA}{L} \\[1em]
        -\dfrac{EA}{L} & \dfrac{EA}{L}
    \end{bmatrix}
\end{dmath*}

Gives the full stiffness matrix:

\begin{dmath*}
    \left[K\right] =
    \begin{bmatrix}
        \dfrac{EA}{L}   & 0                     & 0                     & -\dfrac{EA}{L}    & 0                     & 0 \\[1em]
        0               & \dfrac{12EI}{L^{3}}   & \dfrac{6EI}{L^{2}}    & 0                 & -\dfrac{12EI}{L^{3}}  & \dfrac{6EI}{L^{2}} \\[1em]
        0               & \dfrac{6EI}{L^{2}}    & \dfrac{4EI}{L}        & 0                 & -\dfrac{6EI}{L^{2}}   & \dfrac{2EI}{L} \\[1em]
        -\dfrac{EA}{L}  & 0                     & 0                     & \dfrac{EA}{L}     & 0                     & 0 \\[1em]
        0               & -\dfrac{12EI}{L^{3}}  & -\dfrac{6EI}{L^{2}}   & 0                 & \dfrac{12EI}{L^{3}}   & -\dfrac{6EI}{L^{2}} \\[1em]
        0               & \dfrac{6EI}{L^{2}}    & \dfrac{2EI}{L}        & 0                 & -\dfrac{6EI}{L^{2}}   & \dfrac{4EI}{L}
    \end{bmatrix}
\end{dmath*}

\end{document}